\begin{abstract}


Das Gesetz zur Förderung des elektronischen Rechtsverkehrs mit den Gerichten vom 10.10.2013 erlaubt ab 01.01.2022 als Kommunikationsweg zu den Gerichten einzig den elektronischen Rechtsverkehr.

Für Rechtsanwälte bedeutet dies, dass Schriftsätze bei Gericht ausschließlich auf elektronischem Weg eingereicht werden dürfen. Ein erster Schritt beinhaltet die Einführung besonderer elektronischer Anwaltspostfächer durch die Bundesrechtsanwaltskammer, welche bereits ab dem 01.01.2016 genutzt werden sollen.

Der Beitrag befasst sich mit den daraus resultierenden Konsequenzen sowohl im Bereich der Kommunikationswege zwischen Anwälten und Gerichten als auch in der internen Organisation einer Anwaltskanzlei. Insbesondere sollen die rechtlichen Anforderungen und die daraus ableitbaren technischen Herausforderungen diskutiert und Lösungsansätze erarbeitet werden.
\keywords{ERV-Gesetz, Elektronische Akte, Besonderes elektronisches Anwaltspostfach, Sicherheit und Datenschutz, Integrität, Authentizierung, Autorisierung, Digitale Signaturen, Anwaltskanzlei, Gericht}
\end{abstract}
