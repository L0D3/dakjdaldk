
\section{Auswirkungen der elektronischen Akte auf die Anwaltskanzlei}
In diesem Kapitel werden die Auswirkungen des Gesetzes auf die Anwaltskanzleien und die daraus resultierenden technischen Herausforderungen beschrieben. Dazu wird zuerst anhand eines typischen Ablaufs in einer Kanzlei kurz dargestellt, welche Interaktionen mit einer Akte in einem Prozess durchlaufen werden. Darauf folgend werden rechtliche Anforderungen definiert, die an den Rechtsverkehr und die elektronische Dokumente gestellt werden.
\subsection{Vorgang}
Normalerweise wird von einem Mitarbeiter die Post empfangen, und die enthaltenen Dokumente in die entsprechenden Akten einsortiert. Dem Anwalt werden die aktuellen Akten in sein Postfach abgelegt. Dabei sind die neusten Informationen noch nicht abgeheftet. Nach Erhalt der Akte kann der Anwalt mit seiner Arbeit beginnen. Oftmals werden die Briefe diktiert und von einem Mitarbeiter geschrieben. Dabei ist ein nicht zu unterschätzender Aufwand die Formatierung der Briefe so wie die Ergänzung der Kennnummern.
Die Kommunikation der Anwälte mit den Mandanten erfolgt im Regelfall über Postverkehr. Nach Absprache ist auch die Nutzung von Emails denkbar.
Während der Bearbeitung eines Prozesses unterscheidet der Anwalt zwischen Hand und Gerichtsakte. Dabei beinhaltet die Gerichtsakte die Dokumente, welche vom Anwalt an das Gericht gesendet werden und ist von dem Mandanten zu jeder Zeit einsehbar. Die Handakte übergibt der Anwalt dem Mandanten nach Ablauf des Prozesses nach seiner Bezahlung.
\subsection{Anforderungen}
Um Anforderungen herzuleiten soll ähnlich wie in \textcite{eakten-anforderungen}\footfullcite{eakten-anforderungen}
	  

\subsection{Besonderes elektronisches Anwaltspostfach § 31a Brao}
Ab 01.01.2016 muss von der Bundesrechtsanwaltskammer für jeden Rechtsanwalt ein elektronisches Anwaltspostfach bereitgestellt werden.
\subsubsection*{Erreichbarkeit}
Das elektronische Anwaltspostfach muss für jeden Rechtsanwalt erreichbar sein und darf nicht von diesem ignoriert werden. Elektronische Zustellungen werden zwar nur gegen Empfangsbekenntnis erfolgen. Jedoch ist der Anwalt zu dieser nach § 14 BORA verpflichtet. Die Empfangsbekenntnis kann bis 01.01.2018 auch auf altmodischen Wege erfolgen. Danach muss es auf elektronischen Wege in strukturierte maschinenlesbarer Form stattfinden.
\subsubsection*{Zugangssicherung}
Um die Zugangssicherung zu gewährleisten, darf der Zugang nur nach Authentifizierung durch zwei von einander unabhängigen Sicherungsverfahren erfolgen. Als geeignete Sicherungsmittel sind z.B. eine Chipkarte sowie die Eingabe eines PIN denkbar.  
\subsubsection*{Zugangsberechtigung}
Gemäß § 31a Abs. 2 BRAO können unterschiedliche Zugangsberechtigungen für Rechtsanwälte und für andere Personen eingerichtet werden. So kann in Kanzleien mit mehreren Anwälten der Posteingang bei einem/r Sekretär/in erfolgen und der oft etablierte Arbeitsablauf kann somit beibehalten werden.  Je nach Signatur des Dokuments(siehe Abschnitt Signatur) kann auch die Versendung an das Gericht von dem/r Sekretär/in übernommen werden.

\subsubsection*{Barrierefreiheit}
Nach § 31a Abs. 1 Satz 1 BRAO soll das besondere elektronische Anwaltspostfach barrierefrei ausgestaltet werden. Das Bundesministerium der Justiz wird durch § 31b BRAO ermächtigt, die Einzelheiten zur Barrierefreiheit durch Rechtsverordnung zu regeln.
\subsection{Elektronische Dokumente}
\subsubsection*{Erzeugung}
Jede Kanzlei muss bis zum Inkrafttreten der Nutzungspflicht am 1. Januar 2022 Schriftsätze Anlagen und sonstige Dokumente in elektronischer Form zu erstellen. Aus Sicherheitsgründen (siehe Anforderungen Revisionssicherheit, Authentizität \& Integrität) ist es empfehlenswert die Dokumente im pdf-Format einzureichen.  Papierdokumente von Mandanten können eingescannt und in PDF-Form gespeichert werden


\subsubsection*{Signatur}
Signatur durch besonderes elektronisches Anwaltspostfach
Dateisignatur am Dokument selbst angebracht
Container-Signatur
\subsubsection*{Versand}
Der Versand elektronischer Dokumente an das Gericht ist normalerweise an bestimmte Fristen gebunden.
Die Führung der Akten in der Kanzlei kann sowohl elektronisch als auch in Papierform erfolgen und ist dem Anwalt freigestellt. Die elektronische Authentifizierung eines einzureichendem Dokumentes muss zwingend durch den Anwalt passieren. (vgl. Zugangsberechtigung). Nach der Authentifizierung kann der Versand von einen anderen Kanzleimitarbeiter übernommen werden. Die Übermittlung des Dokuments muss, wie in Abschnitt elektronische Übermittlungswege beschrieben, auf einen sicheren Übermittlungsweg erfolgen. Dies ist z.B. durch die Nutzung des elektronischen Anwaltspostfach gegeben.
\subsubsection*{Barrierefreiheit}

\subsubsection*{Vertraulichkeit}
Nur Menschen mit den notwendigen Berechtigungen dürfen Zugriff auf die Information erhalten. Dies beinhaltet sowohl die Einsicht einzelner Dokumente als auch die Übersicht über die vorhandenen Informationen. Berechtigungen müssen sich je nach Aktenbestand unterscheiden und können sich über einen Zeitraum ändern. Außerdem muss gegebenenfalls zwischen Lese und Bearbeitungsrechten unterschieden werden.
\subsubsection*{Authentizität und Integrität}
Der Autor eines Dokumentes muss klar erkennbar sein. Dabei darf es nicht möglich sein sich als andere Person auszugeben. Ein Dokument darf nur verändert werden, wenn die notwendigen Berechtigungen vorliegen. 
\subsubsection*{Revisionssicherheit}
Falls ein Dokument über einen Zeitraum verändert wurde, muss die Historie der Veränderungen einsehbar sein. Dabei sollte für jede Änderung Zeitpunkt und Autor der Änderung vorliegen. Des weiteren sollten vergangene Änderungen gegebenenfalls rückgängig gemacht werden können.
\subsubsection*{Verbindlichkeit}
Wie im vorigen Abstritt beschrieben muss der Autor eines Dokumentes zu jedem Zeitpunkt eindeutig identifiziert werden können. Daraus folgt, dass es nicht möglich ist die Autorenschaft eines Dokumentes abzustreiten.
\subsubsection*{Verfügbarkeit}
So wie die notwendigen Berechtigungen vorliegen, sollte ein Dokument zu jedem Zeitpunkt einsehbar sein.
\subsubsection*{Schutz personenbezogener Daten}
Trotz der gegebenen Informationen soll die elektronische Akte nicht genutzt werden können um die Arbeitszeiten zu überwachen. So sollen z.B. die Änderungszeitpunkte der Dokumente durch die Richter verborgen bleiben. Diese Anforderung steht im direkten Konflikt mit der Revisionssicherheit.
\subsubsection*{Langzeitarchivierung}
Die obengenannten Anforderungen müssen von der erstmaligen Anlage an erfüllt und bis zum Abschluss des Verfahrens gegeben werden.

