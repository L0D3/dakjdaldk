
\section{Auswirkungen der elektronischen Akte auf die Anwaltskanzlei}
In diesem Kapitel werden die Auswirkungen des Gesetzes auf die Anwaltskanzleien und die daraus resultierenden technischen Herausforderungen beschrieben. Dazu wird zuerst anhand eines typischen Ablaufs in einer Kanzlei kurz dargestellt, welche Interaktionen mit einer Akte während einer anwaltlichen Rechtsberatung erfolgen. Darauf folgend werden rechtliche Anforderungen definiert, die an den Rechtsverkehr und die elektronische Dokumente gestellt werden müssen.
\subsection{Anwaltliche Rechtsberatung}
Zunächst wird – nachdem das Mandatsverhältnis begründet wurde – eine Akte angelegt, sei es physisch oder elektronisch oder beides. Bei der Aktenanlage werden der Akte bereits bestimmte Daten zugewiesen, nämlich insbesondere Name und Anschrift des Mandanten und des Gegners, Kontaktmöglichkeiten, Bezeichnung der Angelegenheit, Verfahrensstand, Art der Beratung (außergerichtlich, gerichtlich 1. Instand, 2. Instanz) sowie ggf. kanzleiinterne Daten wie zum Beispiel Name des Mandatsführers, des Sachbearbeiters und Informationen zu Abrechnung (nach Vergütungsvereinbarung, nach Gebührenstreitwert RVG oder Übernahme Kosten Rechtschutzversicherung bzw. Prozesskostenhilfe). Insbesondere die Adressdaten des Mandanten und des Gegners sind relevant, da es erforderlich wird, in einem frühen Stadium eine Prüfung durchzuführen, ob der Gegner in dieser Angelegenheit bereits durch einen Rechtsanwalt der Kanzlei vertreten wurde oder noch vertreten wird. In diesem Fall, kann das Mandat nicht übernommen werden, da die Gefahr eines Interessenkonflikts besteht. Die Aktenanlage erfolgt üblicherweise durch das Sekretariat auf Veranlassung durch den sachbearbeitenden Rechtsanwalt.
 
Nach erfolgter Aktenanlage wird die Akte dem Rechtsanwalt zur Bearbeitung vorgelegt. Dieser prüft Sachverhalt und Rechtslage und wird je nach Ergebnis der Prüfung entsprechendes veranlassen, also meist Schriftsätze an den Mandanten, den Gegner oder im Falle eines Gerichtsverfahrens an das Gericht abdiktieren, die sodann im Sekretariat geschrieben werden oder diese selbst verfassen. Von jedem Schriftsatz an den Gegner oder das Gericht erhält der eigene Mandant eine Abschrift. Die Kommunikation der Rechtsanwälte mit den Mandanten erfolgt im Regelfall über Briefverkehr. Nach Absprache ist auch die Nutzung von Emails denkbar. In beiden Fällen besteht ein nicht zu unterschätzender Aufwand in der Formatierung der Schriftsätze, des Einfügens der Adressen der Beteiligten und in der Formulierung des korrekten Rubrums (Bezeichnung Mandant und Gegner) bei Schriftsätzen in einem Gerichtsverfahren.
 
Bei der Aktenführung wird üblicherweise zwischen Gerichts- und Handakte unterschieden. Während die Gerichtsakte allen Schriftverkehr mit dem Gericht enthalten sollte, enthält die Handakte die Kommunikation mit dem Mandanten, sowie Aufzeichnungen, Entwürfe und interne Prüfergebnisse des Rechtsanwalts. Das Führen der Akte, also die Zuordnung der Unterlagen zu Gerichts- oder Handakte erledigt meist der Rechtsanwalt. Der Unterschied besteht darin, dass der Rechtsanwalt die eigenen Aufzeichnungen und Prüfungen bei Beendigung des Mandatsverhältnisses zurückhalten darf bis sein Honorar vollständig gezahlt wurde. Außerdem ist es hilfreich bei der Prozessführung, wenn die eigene Akte möglichst vollständig der Akte, die dem Gericht vorliegt, entspricht.
 
Sofern in einer laufenden Angelegenheit Post in der Kanzlei eingeht, wird die Post von dem Sekretariat gesichtet und der jeweiligen Akte zugeordnet und sodann dem Sachbearbeiter vorgelegt. Wenn eine elektronische Akte verwendet wird, wird vor der Vorlage beim Sachbearbeiter der Posteingang eingescannt. Schriftsätze die von Gericht oder dem Gegner kommen, werden dem Mandanten zur Kenntnis übersandt.

\subsection{Anforderungen}
Durch die Einführung des elektronischen Rechtsverkehrs am 01.01.2022 wird die Kommunikation zwischen Anwalt und Gericht ausschließlich auf elektronischem Wege möglich sein. Für die interne Organisation - insbesondere die interne Aktenverwaltung - sowie die Kommunikation zwischen Anwalt und Mandant gibt es keine rechtliche Verordnung. Trotzdem ist aus Effizienzgründen sinnvoll, langfristig komplett auf elektronische Datenspeicherung sowie Kommunikation umzusteigen. Zudem bisher bereits viele Kanzleien, sowohl eine digitale Aktensicherung sowie in Papierform verwenden, und auch die Kommunikation auf elektronischen Wege, wie z.B. per Email oder via direkten Zugriff auf ein digitales Laufwerk erfolgt. 
\subsubsection{Elektronische Aktenführung}\hspace*{\fill} \\
Für die elektronische Aktenverwaltung in einer Kanzlei lassen sich Sicherheitskriterien herleiten, entsprechend der Anforderungen an die E-Akte nach.
\textcite{eakten-anforderungen} wählen die klassischen Schutzziele der It-Sicherheit als Grundlage und konkretisieren diese für den Anwendungsfall der Zivilprozessakte.
\begin{itemize}
\item \textit{Vertraulichkeit:}
Nur Menschen mit den notwendigen Berechtigungen dürfen Zugriff auf die Information erhalten. Dies beinhaltet sowohl die Einsicht einzelner Dokumente als auch die Übersicht über die vorhandenen Informationen. Berechtigungen müssen sich je nach Aktenbestand unterscheiden und können sich über einen Zeitraum ändern. Außerdem muss gegebenenfalls zwischen Lese und Bearbeitungsrechten unterschieden werden.
\item \textit{Authentizität und Integrität:}
Der Autor eines Dokumentes muss klar erkennbar sein. Dabei darf es nicht möglich sein sich als andere Person auszugeben. Ein Dokument darf nur verändert werden, wenn die notwendigen Berechtigungen vorliegen. 
\item \textit{Revisionssicherheit:}
Falls ein Dokument über einen Zeitraum verändert wurde, muss die Historie der Veränderungen einsehbar sein. Dabei sollte für jede Änderung Zeitpunkt und Autor der Änderung vorliegen. Des weiteren sollten vergangene Änderungen gegebenenfalls rückgängig gemacht werden können.
\item \textit{Verbindlichkeit:}
Wie im vorigen Abstritt beschrieben muss der Autor eines Dokumentes zu jedem Zeitpunkt eindeutig identifiziert werden können. Daraus folgt, dass es nicht möglich ist die Autorenschaft eines Dokumentes abzustreiten.
\item \textit{Verfügbarkeit:}
So wie die notwendigen Berechtigungen vorliegen, sollte ein Dokument zu jedem Zeitpunkt einsehbar sein.
\end{itemize}
Sie erweitern die oben genannten Schutzziele um:
\begin{itemize}
\item \textit{Schutz personenbezogener Daten:}
Trotz der gegebenen Informationen soll die elektronische Akte nicht genutzt werden können um die Arbeitszeiten zu überwachen. So sollen z.B. die Änderungszeitpunkte der Dokumente durch die Richter verborgen bleiben. Diese Anforderung steht im direkten Konflikt mit der Revisionssicherheit.
\item \textit{Langzeitarchivierung:}
Die Ziele müssen von der erstmaligen Anlage der E-Akte über den rechtskräftigen Abschluss hinaus erreicht werden.
\end{itemize}
\textcite{rechtsvekehranwaltskanzlei} benennt weitere Anforderungen an elektronische Dokumente innerhalb der Anwaltskanzlei:
\begin{itemize}
\item \textit{Erzeugung:}
Jede Kanzlei muss bis zum Inkrafttreten der Nutzungspflicht am 1. Januar 2022 Schriftsätze Anlagen und sonstige Dokumente in elektronischer Form zu erstellen. Aus Sicherheitsgründen (siehe Anforderungen Revisionssicherheit, Authentizität \& Integrität) ist es empfehlenswert die Dokumente im pdf-Format einzureichen.  Papierdokumente von Mandanten können eingescannt und in PDF-Form gespeichert werden
\item \textit{Signatur:}
Signatur durch besonderes elektronisches Anwaltspostfach
Dateisignatur am Dokument selbst angebracht
Container-Signatur
\item \textit{Versand:}
Der Versand elektronischer Dokumente an das Gericht ist normalerweise an bestimmte Fristen gebunden.
Die Führung der Akten in der Kanzlei kann sowohl elektronisch als auch in Papierform erfolgen und ist dem Anwalt freigestellt. Die elektronische Authentifizierung eines einzureichendem Dokumentes muss zwingend durch den Anwalt passieren. (vgl. Zugangsberechtigung). Nach der Authentifizierung kann der Versand von einen anderen Kanzleimitarbeiter übernommen werden. Die Übermittlung des Dokuments muss, wie in Abschnitt elektronische Übermittlungswege beschrieben, auf einen sicheren Übermittlungsweg erfolgen. Dies ist z.B. durch die Nutzung des elektronischen Anwaltspostfach gegeben.
\end{itemize}
	  

\subsubsection{Besonderes elektronisches Anwaltspostfach § 31a Brao}\hspace*{\fill} \\
Für die sichere elektronische Kommunikation zwischen Anwälten und Gerichten soll ab dem 01.01.2016 das besondere elektronische Anwaltspostfach bereitgestellt werden. \textcite{rechtsvekehranwaltskanzlei} definiert die hierzu folgenden Anforderungen:
\begin{itemize}
\item \textit{Erreichbarkeit:}
Das elektronische Anwaltspostfach muss für jeden Rechtsanwalt erreichbar sein und darf nicht von diesem ignoriert werden. Elektronische Zustellungen werden zwar nur gegen Empfangsbekenntnis erfolgen. Jedoch ist der Anwalt zu dieser nach § 14 BORA verpflichtet. Die Empfangsbekenntnis kann bis 01.01.2018 auch auf altmodischen Wege erfolgen. Danach muss es auf elektronischen Wege in strukturierte maschinenlesbarer Form stattfinden.
\item \textit{Zugangssicherung:}
Um die Zugangssicherung zu gewährleisten, darf der Zugang nur nach Authentifizierung durch zwei von einander unabhängigen Sicherungsverfahren erfolgen. Als geeignete Sicherungsmittel sind z.B. eine Chipkarte sowie die Eingabe eines PIN denkbar.  
\item \textit{Zugangsberechtigung:}
Gemäß § 31a Abs. 2 BRAO können unterschiedliche Zugangsberechtigungen für Rechtsanwälte und für andere Personen eingerichtet werden. So kann in Kanzleien mit mehreren Anwälten der Posteingang bei einem/r Sekretär/in erfolgen und der oft etablierte Arbeitsablauf kann somit beibehalten werden.  Je nach Signatur des Dokuments(siehe Abschnitt Signatur) kann auch die Versendung an das Gericht von dem/r Sekretär/in übernommen werden.
\end{itemize}
Neben den Sicherheitsanforderungen ergeben sich für das Anwaltspostfach noch einige Usability- und verwaltungstechnischen Anforderungen. Diese haben allerdings aus technischer Sicht keine kritischen Herausforderungen. Sie werden in Kapitel 5 bei der ausführlichen Beschreibung des Anwaltspostfach noch genannt.