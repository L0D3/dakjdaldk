\section{De-Mail}
Der Staat reagierte auf die geringe Akzeptanz des EGVP mit dem De-Mail-Gesetz.\textcite{bea:demailgesetz} \footfullcite{bea:demailgesetz} Damit sollte die Grundlage für den verbindlichen, sicheren, vertraulichen und nachweisbaren Versand elektronischer Dokumente und Nachrichten gelegt werden.\textcite{bea:demail} \footfullcite{bea:demail} \\
Es gibt verschiedene De-Mail-Dienste, die dem Verbraucher zur Verfügung gestellt werden. Dazu gehören der Versand- und Postfachdienst, der De-Safe und der Dienst für zuverlässigen Identitätsnachweis, De-Ident. \\
Die De-Mail kann von Jedermann benutzt werden. Sie wird nicht vom Staat selbst bereitgestellt, stattdessen wurden zertifizierte Unternehmen beauftragt. Durch die Implementierung internationaler Standards können diese Unternehmen ihren Kunden eine sichere und rechtsverbindliche Kommunikation anbieten. Die De-Mail verfügt dabei über wichtige Eigenschaften, die eine herkömmliche E-Mail nicht hat:
\begin{itemize}
	\item Durch den De-Ident-Dienst können die Identitäten von Absender und Empfänger eindeutig nachgewiesen und nicht gefälscht werden. 
	\item Nutzer des De-Mail-Versanddienstes können Nachrichten als Einschreiben verschicken. Dadurch erhalten sie eine qualifiziert signierte Bestätigung, wann die Nachricht abgeschickt und wann sie im Postfach des Empfängers angekommen ist.
	\item Der Versand von Nachrichten wird ausschließlich über verschlüsselte Kanäle getätigt. Im Postfach des Empfängers werden die Nachrichten nur verschlüsselt abgespeichert (De-Safe-Dienst). Dadurch sollen sie zu keiner Zeit von Dritten im Klartest gelesen oder gefälscht werden können.
\end{itemize}

Am 1. August 2013 ist das E-Government-Gesetz \textcite{bea:e-government}\footfullcite{bea:e-government} in Kraft getretenen. Unter bestimmten Voraussetzungen ist es nun auch möglich die in Verwaltungsakten erforderliche Schriftform durch den Versand einer De-Mail zu ersetzen. \\
Der De-Mail-Dienst sollte vor allem aufgrund der hohen Sicherheitsstandard für Bürgen und Behörden attraktiv gemacht werden. Allerdings weist der Dienst erhöhte Mängel auf:
''Die hohe Relevanz und Vertraulichkeit der per De-Mail versendeten Dokumente erhöht die Attraktivität der De-Mail-Server als Angriffsziele. Dem entsprechend zu erwartenden Angriffsvolumen steht aufgrund der mangelnden Ende-zu-Ende-Verschlüsselung kein adäquates Sicherheitskonzept entgegen.'' \textcite{bea:demail:ccc}\footfullcite{bea:demail:ccc} Der De-Mail-Versanddienst verwendet authentisierte und verschlüsselte Kommunikationskanäle zwischen Nutzern und Anbietern, als auch zwischen Anbietern untereinander. Nachrichten werden erst auf Anbieter-Seite des Senders elektronisch signiert und verschlüsselt und danach zum Empfänger-Anbieter versandt. Dieser muss die Nachricht allerdings zuerst entschlüsseln, um die Integrität des Senders zu überprüfen. Danach wird die Nachricht abermals verschlüsselt und ins Empfänger-Postfach gelegt. Dadurch entstehen die hohen Sicherheitsmängel seitens der Anbieter. Aber auch die Tatsache, dass Nachrichten erst auf Anbieter-Seite vom Anbieter selbst signiert werden, erscheint falsch. Der Nutzer hat keinerlei Kontrolle und muss seinem Anbieter vertrauen, dass erstens diesem keine Fehler passieren und zweitens seine elektronische Signatur nicht missbraucht wird. \\
Weiterhin muss lediglich beim Erstellen des De-Mail-Accounts ein Identitätsnachweis - meist in Form eines amtlichen Lichtbildausweises oder der eID-Funktion des Personalausweises, erbracht werden. Anfänglich wurde danach nur Nutzerkennung und Passwort benötigt, um sich ins Postfach einzuloggen. Diesem Sicherheitsmangel sollte durch §4 (2) des De-Mail-Gesetzes vom Bundesamt für Sicherheit in der Informationstechnik (kurz BSI) durch das mTan-Verfahren zur verstärkten Sicherheit entgegengewirkt werden. Diese Neuregelung sieht einen zweiten Authentifizierungsweg zum Einloggen vor, wie zum Beispiel der Eingabe eines Authentifizierungscode der per SMS auf die hinterlegte Mobilnummer entsandt wird. Aber auch das mTAN-Verfahren ist nicht sicher. Bislang wird dieses Konzept bei Banken verwendet und kann beispielsweise durch die Infektion des Telefons, Abhören von SMS-Inhalten oder durch Impersonierung, sprich Kopieren der SIM-Karte, angegriffen werden. \textcite{bea:demail:brokenbydesign}\footfullcite{bea:demail:brokenbydesign} \\
Ein weiterer Nachteil ist die fehlende Integration des Dienstes für den Versand an Postfächer des EGVP und der geringe Datenschutz \textcite{bea:demail}:
\begin{itemize}
	\item Laut §112 des Telekommunikationsgesetzes kann die Identität hinter einem De-Mail-Nutzerkonto von etwa 250 registrierten Behörden online abgerufen werden.
	\item Laut §113 des Telekommunikationsgesetzes sind die persönlichen Daten des Nutzers für eine Vielzahl von Sicherheitsbehörden und Geheimdienstes ohne richterliche Anordnung einsehbar.
\end{itemize}

Auf dem 30. Chaos Communication Congress stellte der Sicherheitsanalyst Linus Neumann seine Analyse des De-Mail-Dienstes mit dem Titel ''Bullshit made in Germany'' vor. Er formulierte treffend: ''De-Mail [sei] absichtlich unsicher gebaut, um deutschen Diensten zu ermöglichen, deutsche Bürger auszuspähen.''\textcite{bea:demail:bullshit}\footfullcite{bea:demail:bullshit}
 