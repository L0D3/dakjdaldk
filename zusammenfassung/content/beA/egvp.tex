\section{EGVP}
Das \textbf{elektronische Gerichts- und Verwaltungspostfach} (kurz EGVP) ist ein Postfach für die Justiz, das von der öffentlichen Verwaltung, Bürgern, Unternehmen, Inkassogesellschaften, Rechtsanwälten, Notaren oder Gerichtsvollziehern über einen speziellen EGVP-Classic-Bürger-Client erreicht werden kann, um Dokumente elektronisch einzureichen. \\
Nachrichten, die über den EGVP-Client versendet werden, unterliegen dem OSCI-Sicherheitsstandard und sind damit auch, im Gegensatz zum Standardversand von Nachrichten bei der De-Mail, Ende-zu-Ende verschlüsselt. Mit der Einführung des besonderen elektronischen Anwalts- (kurz beA) beziehungsweise Notarpostfachs, wird der EGVP-Client obsolet. Ein Grund dafür ist sicherlich die geringe Akzeptanz des eher sperrigen und fehleranfälligen Clients. Ein anderer scheint gewiss die schlechte Integrität des EGVPs zu sein: erst in sechs Bundesländer kommt das EGVP zum Einsatz, allerdings auch bei diesen nicht an jedem Gericht. ''[...] Nur in Berlin, Brandenburg, Bremen, Hessen[, Rheinland-Pfalz (Verwaltungs-, Sozial- und Finanzgerichtsbarkeit)] und Sachsen können Schriftsätze und Klagen über das EGVP bei ordentlichen und besonderen Gerichtsbarkeiten eingereicht werden.
In manchen Bundesländern wird die Technik nur an ganz bestimmten Gerichten eingesetzt. In Hamburg ist es das Finanz-, in Schleswig-Holstein sind es die Arbeitsgerichte. Bis tatsächlich alle Gerichte in ganz Deutschland über die notwendige Technik für den neuen Standard verfügen werden, hat der Gesetzgeber eine Frist bis spätestens zum Jahr 2020 gelassen. Bis dahin jedoch ist das elektronische Anwaltspostfach, welches das EGVP ersetzen soll, ein Programm mit vielen Sendern – aber ein Großteil der wichtigsten Empfänger fehlt.'' \textcite{bea:egvp:landkarte}\footfullcite{bea:egvp:landkarte} \\
\\
Das EGVP findet eher wenig Verwendung, weshalb es vom besonderen elektronischen Anwalts- bzw. Notarpostfachs abgelöst wird. Dies ist im Gesetz zur Förderung des elektronischen Rechtsverkehrs festgelegt. Jedoch wurde von der Bund-Länder-Kommission für Informationstechnik (BLK) in der Justiz beschlossen, dass die Infrastrukturkomponenten für die Kommunikation, soll heißen die Postfächer und die Adressen im Verzeichnisdienst S.A.F.E, für Nutzer die weder Rechtsanwalt noch Notar sind, weiterhin unverändert zur Verfügung stehen werden. \textcite{bea:egvp}\footfullcite{bea:egvp} \\
Jedoch wird der EGVP-Bürger-Client, das bedeutet die Software zum Kommunizieren mit dem EGVP, am 01.01.2016 offiziell eingestellt, steht den Nutzern jedoch noch bis zum 30.09.2016 - allerdings ab dem 01.04.2016 ohne Support, zur Verfügung. Danach müssen Nutzer auf Software von Drittherstellern zurückgreifen. Die beschlossene Übergangsfrist, vom 01.01.2016 bis zum 01.04.2016, soll dabei den Rechtsanwälten und Notaren zugutekommen. Sie stellt einen Zeitraum dar, in dem das besondere elektronische Anwalts- bzw. Notarpostfach und der EGVP-Bürger-Client parallel betrieben werden können, so dass der laufende Kanzleibetrieb nicht beeinträchtigt wird.