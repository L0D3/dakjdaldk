\section{Fazit und Ausblick}
Im ersten Teil der Arbeit wurde die gesetzliche Lage bzgl. der elektronischen Rechtsverkehr vorgestellt. Danach wurde die aktuelle Situation in einer Anwaltskanzlei beschrieben und die sich durch die gesetzliche �nderungen ergebenen neuen Anforderungen - vor allem im Bezug auf Datensicherheit und Integrit�t - aufgezeigt. Anschlie�end wurden diese Anforderungen auf technische Ebene �berf�hrt und verschiedene Algorithmen und Konzepte beschrieben, welche zur L�sung genutzt werden k�nnten. In den folgenden Kapiteln wurden verschiedene L�sungsans�tze pr�sentiert und insbesondere auf die geplante Umsetzung des beA n�her eingegangen.

   
- fraglich ob notwendig
- kann arbeit erleichtern
- strenger zeitplan
- kostet erstmal
- kurzfristig schwierig umzusetzen, mehr kosten, langfristig bestimmt n�tzlich
-anw�lte benutzen kanzleisoftware die funktioniert, eventuell mehr features kann
- von technischer ebene umsetzbar, die technischen algorithmen existieren, allerdings infrastruktur aufzusetzen ist schwierig, ben�tigt eingew�hnungszeit. f�r kleine kanzleien lohnt sich die umstellung vielleicht nicht. Problematik dass mitarbeiter das nicht nutzen k�nnen. anwalt arbeit �bernehmen muss, die vorher vom sekret�riat getragen wurde .Bei den anw�lten eventuelle gro�e umstellung im arbeitsablauf. Zwar werden oft schon elektronische akten benutzt, allerding meistens auch papier akte. fraglich ob das bea viel bringt wenn der andere kommunilkation mit mandanten extern geregelt werden muss. M�sste eventuell auch das bieten. 
-  