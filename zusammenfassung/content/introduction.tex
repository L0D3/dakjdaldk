\section{Einleitung}

\subsection{Motivation}
Der digitale Fortschritt ist in der heutigen Welt nicht mehr wegzudenken. Dies beinhaltet auch die Digitalisierung innerhalb der Justiz. Durch das Gesetz zur Förderung des elektronischen Rechtsverkehrs mit den Gerichten vom 10.10.2013 erhält diese Entwicklung einen weiteren deutlichen Schub. In naher Zukunft soll die Kommunikation mit den Gerichten ausschließlich elektronisch ablaufen. Dazu soll bis zum 01.01.2016 ein besonderes elektronisches Anwaltspostfach eingeführt werden und bis zum 01.01.2022 soll die Übermittlung der vorbereitenden Schriftsätze sowie deren Anlagen zwischen Rechtsanwälten und den Gerichten nur noch auf elektronischen Wege stattfinden. 
Damit das Gesetz erfolgreich umgesetzt werden kann, bedarf es Anpassungen sowohl in den Kommunikationswegen zwischen Anwälten und Gerichten als auch in der internen Organisation in der Anwaltskanzlei. Insbesondere ist eine Wechsel auf eine digitale Datenverwaltung zwar nicht rechtlich gefordert, aber allein aus Effizienzgründen sinnvoll. Dabei ergeben sich technische und rechtliche Bedingungen, welche miteinbezogen werden müssen.  

\subsection{Zielstellung}
Das Ziel dieser Arbeit ist die Ausarbeitung der technischen und rechtlichen Aspekte der elektronischen Akte innerhalb der Anwaltskanzlei. Insbesondere sollen:
\begin{itemize}
\item die sich neu ergebenen rechtlichen Herausforderungen während der Kommunikation zwischen Anwalt und Client sowie Anwalt und Gericht als auch innerhalb der internen Organisation definiert werden.  Dabei soll vor allem das neue elektronische Anwaltspostfach näher beleuchtet werden.
\item aus den rechtlichen Aspekten, die technischen Problemstellungen abgeleitet und unter zu Hilfenahme aktueller Literatur verschiedene Lösungsansätze herausgefiltert werden.
\item  Vorschläge gegeben werden, wie durch Einführung der E-Akte die Arbeitsabläufe in der Anwaltschaft und mit dem Mandanten sowie vor Gericht beschleunigt werden können.
\end{itemize} 

\subsection{Gliederung}
Im ersten Teil der Arbeit (Kapitel 2) wird die aktuelle Gesetzeslage bzgl. des Gesetz zur Förderung des elektronischen Rechtsverkehrs mit den Gerichten vorgestellt. Dabei werden insbesondere die Begriffe  Elektronisches Übermittlungswege, Elektronische Dokumente, Elektronische Formulare sowie das Elektronische Schutzschriftenregister näher erörtert. Kapitel 3 beschreibt die Auswirkungen des Gesetzes auf die Anwaltskanzleien und die daraus resultierenden technischen Herausforderungen. In Kapitel 4 wird der technische Hintergrund geschaffen, um die in Kapitel 3 erörterten Anforderungen technisch zu diskutieren. Dabei werden unter anderem aktuelle Verfahren zur Authentifizierung, Autorisierung und zur Erstellen von digitalen Signaturen vorgestellt. Kapitel 5-7 befassen sich mit der visuellen und technischen Umsetzung des Anwaltspostfachs. Dabei werden in Kapitel 5 und 6 die ersten beiden Versuche zur sicheren Kommunikation mit der Justiz, das momentan verwendete EGVP und die ''sichere'' De-Mail, erläutert. Kapitel 7 gibt genaueren Einblick in das besondere elektronische Anwaltspostfach und stellt darin die neusten veröffentlichten Informationen vom Juni 2015 vor. Abschließend werden in Kapitel 8 die Ergebnisse der vorherigen Kapitel zusammengefasst und Ergänzungen sowie ein Ausblick gegeben.

