\section{Fazit und Ausblick}
Im ersten Teil der Arbeit wurde die gesetzliche Lage bzgl. der elektronischen Rechtsverkehr vorgestellt. Danach wurde die aktuelle Situation in einer Anwaltskanzlei beschrieben und die sich durch die gesetzliche Änderungen ergebenen neuen Anforderungen - vor allem im Bezug auf Datensicherheit und Integrität - aufgezeigt. Anschließend wurden diese Anforderungen auf technische Ebene überführt und verschiedene Algorithmen und Konzepte beschrieben, welche zur Lösung genutzt werden könnten. In den folgenden Kapiteln wurden verschiedene Lösungsansätze präsentiert und insbesondere auf die geplante Umsetzung des beA näher eingegangen.

Aus unserer Sicht kann sich durch die Einführung der E-Akte und der Integration des beA in den Kanzleien sicherlich eine Verbesserung ergeben. Im Falle einer richtig konzipierten Umsetzung kann die Sicherheit erhöht und eventuell auch eine Effizienzsteigerung im Arbeitsablauf in den Kanzleien erzielt werden. Durch die Verwendung geeigneter kryptographischer Algorithmen, wie in Kapitel 4 beschrieben, sollte die Entwicklung nach technischer Kriterien möglich sein. Allerdings sind auch einige Zweifel nicht unbegründet. Obwohl langfristig eventuell eine Zeit-  und Kostenersparnis erreicht werden kann, so werden für die Kanzleien erstmals zusätzliche Kosten für die neue Software fällig. Außerdem müssen Arbeitsabläufe grundlegend  umgestellt werden. In der momentan geplanten Umsetzung, fehlt der Zugriff der Mitarbeiter auf das beA und das Senden und Empfangen der Dokumente muss von dem Anwalt übernommen werden. Insgesamt wird das Auslagern von organisatorischen Arbeiten auf Dritte aus Gründen der Datensicherheit erschwert. Außerdem sieht das beA nur die Kommunikation zwischen Anwalt und Gericht in Betracht. Für die Kommunikation mit de Mandanten bietet es keine Lösung.  
Ein weitere Schwierigkeit ist, dass die meisten wichtige Dokumente als Originale vorliegen müssen. Insofern ist eine Papierakte immer noch notwendig und die E-Akte kann nur parallel geführt werden.
     