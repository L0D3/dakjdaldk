\section{Gesetz zur Förderung des elektronischen Rechtsverkehrs mit den Gerichten}
Das Gesetz zur Förderung des elektronischen Rechtsverkehrs mit den Gerichten ändert die Prozessordnungen und Verfahrensgesetze für die Gerichte grundlegend. Bisher konnten die Länder selbst über die Einführung der elektronischen Verordnungswege entscheiden. Dies wird durch eine bundesweit eintretende Regelung am 01.01.2022 ersetzt, die ausschließlich elektronischen Kommunikationswege für Anwälte und Behörden zu den Gerichten erlaubt. Gleichzeitig wurden an mehreren Stellen Vorschriften zur Barrierefreiheit in die Gesetze eingefügt. In diesem Kapitel werden die wesentlichen gesetzlichen Änderungen vorgestellt. Dazu werden als Quellen im wesentlichen \textcite{Gesetzfoerderungrechtsverkehr}\footfullcite{Gesetzfoerderungrechtsverkehr} und \textcite{carstens2015grundlagen}\footfullcite{carstens2015grundlagen} herangezogen.
\subsection*{Elektronisches Anwaltspostfach}
Nach § 31a der Bundesrechtsanwaltordnung (BRAO) müssen Rechtsanwälte ab dem 1. Januar 2016 für die Gerichte über ein besonderes elektronisches Anwaltpostfach erreichbar sein. Die besonderen elektronischen Anwaltspostfächer werden von der Bundesrechtsanwaltskammer für die Rechtsanwälte eingerichtet. Eine detaillierte Beschreibung des elektronischen Anwaltspostfach folgt in Kapitel 5.
\subsection*{Elektronische Übermittlungswege}
Die Vorschrift des § 130a der Zivilprozessordnung (ZPO) sieht vor, dass vorbereitende Schriftsätze und deren Anlagen, schriftlich einzureichende Anträge und Erklärungen der Parteien sowie schriftlich einzureichende Auskünfte, Aussagen, Gutachten, Übersetzungen und Erklärungen Dritter ab dem 01.01.2018 flächendeckend als elektronisches Dokument bei Gericht eingereicht werden können. Während elektronische Dokumente an das Gericht bisher mit einer qualifizierten elektronischen Signatur nach dem Signaturgesetz versehen sein müssen, wird es zukünftig möglich sein, elektronische Dokumente auch ohne qualifizierte elektronische Signatur zu übermitteln, wenn hierfür einer der nachfolgenden, vom Gesetzgeber im Hinblick auf die die Authentizität und die Integrität des übermittelten elektronischen Dokuments als sicher bezeichneten Übermittlungswege genutzt werden. Als sichere Übermittlungswege benennt das Gesetz in § 130a Abs. 4 ZPO erstens den Postfach und Versanddienst eines De-Mail Kontos, zweitens die Nutzung des elektronischen Anwaltspostfach und der elektronischen Poststelle des Gerichts, drittens den Übermittlungsweg zwischen einem hierfür eingerichteten elektronischen Postfach einer Behörde und der elektronischen Poststelle des Gerichts und viertens sonstige  bundeseinheitliche Übermittlungswege, die durch Rechtsverordnung festgelegt werden.
\subsection*{Elektronische Dokumente}
Mit der flächendeckenden Einführung des elektronisches Rechtsverkehrs wird es möglich werden, sowohl die Schriftsätze der Verfahrensbeteiligten und Erklärungen Dritter als auch gerichtliche Dokumente(Urteile, Beschlüsse, Protokolle,etc..) als elektronisches Dokument zu übermitteln. Nach § 174 Abs. 3 Satz 4 ZPO werden Rechtsanwälte und andere Prozessbevollmächtigte zudem verpflichtet, ab dem 01.01.2018 einen sicheren Zugang im Sinne des § 130a ABS.4 ZPO für Zustellungen elektronischer Dokumente durch das Gericht zu eröffnen. Daher werden elektronische Dokumente des Gerichts die bisherigen Papierdokumente zunehmend ersetzen. § 191a ABs. 3 Satz 1 GVG sieht vor, dass die Texte barrierefrei zugänglich und nutzbar sein müssen. Daneben können auch Fotos und Skizzen beigefügt werden. Für elektronische Dokumente, die an das Gericht übermittelt werden, sieht § 130a Abs.2 Satz 1 ZPO vor, dass sie für die Bearbeitung durch das Gericht geeignet sein müssen. Die technischen Rahmenbedingungen werden durch die Rechtsverordnung festgelegt. Insbesondere soll eine Bearbeitungs- und Suchfunktion der elektronischen Dokumente bereitgestellt werden. Daher soll die Übermittlung elektronischer Dokumente als Scans bzw. Bilder auf zu Beweiszwecken eingescannte Urkunden, Nachweise und Belege begrenzt werden. 
\subsection*{Elektronische Formulare}
Nach § 130c ZPO kann das Bundesministerium der Justiz für die elektronische Kommunikation mit den Gerichten ab dem 1. Juli 2014 durch Rechtsverordnung elektronische Formulare einführen. Die Rechtsverordnung kann bestimmen, dass die in den Formularen enthaltenen Angaben ganz oder teilweise in strukturierter maschinenlesbarer Form zu übermitteln sind. Die Formulare sind auf einer in der Rechtsverordnung zu bestimmenden Kommunikationsplattform im Internet zur Nutzung bereitzustellen.  Die Rechtsverordnung kann bestimmen, dass eine Identifikation des Formularverwenders abweichend von § 130 a Absatz 3 ZPO auch durch Nutzung des elektronischen Identitätsnachweises nach § 18 des Personalausweisgesetzes erfolgen kann. Elektronische Formulare sind nach § 191a Abs. 3 Satz 1 bzw. Satz 3 GVG blinden oder sehbehinderten Personen barrierefrei zugänglich und nutzbar zu machen.
\subsection*{Elektronisches Schutzschriftenregister}
Die Vorschrift des § 945a Abs. 1 ZPO sieht vor, dass die Länder ein zentrales länderübergreifendes elektronisches Register für Schutzschriften führen. Schutzschriften sind vorbeugende Verteidigungsschriftsätze gegen erwartete Anträge auf Arrest oder einstweilige Verfügung . Hierzu regelt die Verordnungsermächtigung in § 945 b ZPO, dass das Bundesministerium der Justiz durch Rechtsverordnung die näheren Bestimmungen über die Barrierefreiheit festlegt.

Auch dieses neue Regelung sieht ab 1.1.2017 eine Nutzungspflicht für Anwälte vor. Nach § 49c BRAO ist der Rechtsanwalt verpflichtet Register ausschließlich über das elektronische Schutzschriftenregister einzureichen.
\newpage